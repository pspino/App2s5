\documentclass[12pt]{article}
\usepackage{array}
\usepackage[utf8]{inputenc}

\begin{document}
\begin{titlepage}
	\centering
	{\LARGE Jacob Fontaine \par
	 \LARGE Philippe Spino \par}
	\vspace{1cm}
	\small Spip2401\par
	\small Fonj\par 
	\vspace{3cm}
	{\Large Rapport App2\par}
	\vspace{5cm}
	{\Large Présente à :\par JEAN LAVOIE\par}
	\vspace{4cm}
	\vfill
% Bottom of the page
	{\large \today\par}
\end{titlepage}

\newpage
\tableofcontents
\newpage
%%\linespacing{1.5}
\section{Introduction.}
Ce présent rapport a pour but de présenter les éléments de conception du système d'acquisition de données(SAD) développé par notre équipe. Le SAD a été conçue à l'aide d'un processeur embarqué LPC1768, soit la coquille de base du SAD. Le contenue de ce rapport explique les ressources mémoire et du processeur utilisées, les algorithmes de gestion des acquisitions, la synchronisation des tâches utilisées, une analyse des problèmes de conception et de déverminage lors du montage du prototype, une description des méthodes concernant la validation des fréquences d'échantillonnages des données et une analyse du prototype le plus optimal. De plus, une discusion sur les temps d'éxecution des codes base pour un inclinomètre. à la demande de la compagnie metrologie numérique.
\section{Utilisation des ressources matérielles.}
\noindent
Pour ce présent prototype deux LCP1768 sont utilisés. Un des microcontrôleur est programmé pour emêttre de façon aléatoire des données soit numériquement ou bien analogiquement. Il est nommé DataSpammer. L'autre microcontrôleur utilise l'interface principale du SAD. Le DataSpammer, est réglé avec une fréquence d'opération mobile, c'est-à-dire qu'elle peut etre changé selon l'utilisateur.
\section{Algorithmes de gestion des acquisitions de données.}
L'algorithme dévéloppé pour l'aquisition des données est divisé en 3 threads différents. Numérique, Analogique et les autres types de réceptions. Le thread pour l'acquisition numérique étant en priorisation haute, l'analogique en priorisation moyenne et le reste en priorisation basse.
\subsection{Numérique}
\noindent
Le thread pour l'acqusition des données numérique vérifie s'il y a un changement de bit sur le front montant des changements des bits. Le temps pris pour qu'un bit puisse changer est de l'ordre des 50 ms. La fréquence d'opération de cette algorithmes est 100ms. Pour ce faire, un thread::wait fû utilisé. cela permet d'être assuré que le thread vas être à une fréquence de 100ms. On considère que les opérations fait par le CPU sont non significatif sur le temps d'opération puisque le CPU opère a une fréquence de 96MHz, soit 1000 fois plus rapidement que les fréquences d'acquisition de données.
\section{Synchronisation des tâches CPU.}
\section{Conception du montage prototype.}
\section{Méthode de validation des fréquences d'échantillonage.}
\section{Optimisation du prototype.}
\section{Analyse de performances des Codes de Métrologie Numérique.}
\section{Conclusion.}
\end{document}